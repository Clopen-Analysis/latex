\appendix
\chapter{Linear algebra}
\section{Normed spaces}
The reader should be familiar with this material before reading Chapter \ref{measureChapter}.

When one first learns what a ``vector" is, they are told that a vector is comprised of a length and a direction.
However, the algebraic definition of a vector space does not satisfy this property; nothing in the definition of a vector space allows one to canonically assign lengths to vectors.
In this section we correct this matter by introducing a notion of length.

We take all vector spaces to be over the real numbers $\RR$ or the complex numbers $\CC$ (preferably the latter).
We let $[0, \infty)$ denote the nonnegative real numbers.

\begin{definition}
A \dfn{seminormed space} is a vector space $V$ equipped with a function
\begin{align*}V &\to [0, \infty)\\
v &\mapsto ||v||,\end{align*}
known as a \dfn{seminorm}, such that for any $v, w \in V$ and $c$ a scalar,
$$||v + w|| \leq ||v|| + ||w||,$$
the \dfn{triangle inequality}, and
$$||cv|| = |c|\cdot||v||.$$
The quantity $||v||$ is called the \dfn{length} of $v$.

A \dfn{normed space} is a seminormed space $V$ such that the only $v \in V$ such that $||v|| = 0$ is $v = 0$.
A seminorm with this property is called a \dfn{norm}.
\end{definition}

Every vector space can be turned into a seminormed space in a trivial way, namely by setting $||v|| = 0$ for every $v$.
However, we willl have no use for this.

\begin{example}
The most important example of a normed space has as its base space $\RR^d$.
We define
$$||(x_1, \dots, x_d)||_2^2 = \sum_{i=1}^d |x_i|^2,$$
thus $||\cdot||_2$ is the \dfn{Euclidean norm} on $\RR^d$.
Then the triangle inequality is just the usual triangle inequality for Euclidean geometry.
\end{example}

\begin{example}
We can define several other norms on $\RR^d$, closely related to the Euclidean norm $||\cdot||_2$. First we let
$$||(x_1, \dots, x_d)||_\infty = \max_{i=1}^d |x_i|.$$
We then define, for any $p \in [1, \infty),$
$$||(x_1, \dots, x_d)||_p^p = \sum_{i=1}^d |x_i|^p.$$
Thus when $p = 2$ we recover the Euclidean norm, when $p = 1$ we recover the ``sum norm" $||(x_1, \dots, x_d)||_1 = \sum_i |x_i|$, and in the limit $p \to \infty$ we obtain the $||\cdot||_\infty$ norm.
\end{example}

A normed space is a metric space in a natural way, namely the distance between two vectors $v,w$ is defined to be $||v - w||$.
Thus we have access to the usual notion of sequences, convergence, etc. for normed spaces; so the equation
$$\lim_{n \to \infty} v_n = w$$
means that for every $\varepsilon > 0$ there is an $N$ such that for every $n \geq N$, $||v_n - w|| < \varepsilon$.
Similarly a sequence of $v_n$ is Cauchy if for every $\varepsilon > 0$ there is an $N$ such that for every $n, n' \geq N$, $||v_n, v_{n'}|| < \varepsilon$.

The notion of convergence makes sense in seminormed spaces, but it is no longer true that the limit of a sequence need be unique.
In fact, consider the seminorm on $\RR^2$ defined by
$$||(x, y)|| = |x|.$$
Let $(x_n, y_n)_n$ be a sequence in $\RR^2$ and $x \in \RR$. Then for every $\lim_n x_n = x$, then for every $y \in \RR$, $\lim_n (x_n, y_n) = (x, y)$.
In the language of point-set topology, seminormed spaces are not Hausdorff (nor do they even satisfy Axiom $T_0$).

We fix the problem with seminormed spaces by observing that it is always possible to turn a seminormed space into a normed space.
\begin{theorem}
\label{existence of normalization}
For every seminormed space $V$ there is a normed space $V'$, called the \dfn{normalization} of $V$, such that:
\begin{enumerate}
\item There is a surjective linear map $\pi: V \to V'$.
\item For every normed space $W$ and every linear map $T: V \to W$ such that for every $v$, $||Tv|| = ||v||$, there is a linear map $T': V/\ker V \to W$ such that $||T'v|| = ||v||$ and the diagram
\begin{equation}
\label{normification universal}
\begin{tikzcd}
V \arrow[rr,"T"] \arrow[dr,"\pi"] && W\\
& \frac{V}{\ker V} \arrow[ur,"T'"]
\end{tikzcd}
\end{equation}
commutes.
\end{enumerate}
\end{theorem}
In the language of category theory, the diagram (\ref{normification universal}) is called the \dfn{universal property of the normalization}.
In category theory one proves that whenever an algebraic object satisfies a universal property, the object is unique up to a unique choice of isomorphism.
Thus the normalization (and the closely related completion that we discuss in Theorem \ref{completion exists}) is unique.

The reader should omit the proof of Theorem \ref{existence of normalization} on first reading.
\begin{proof}[Proof of Theorem \ref{existence of normalization}]
In fact, we define the \dfn{kernel} of a seminormed space $V$ by $\ker V = \{v \in V: ||v|| = 0\}$.
The kernel is a subspace of $V$, so we can take the quotient space $V/\ker V$ and form a short exact sequence
\begin{equation}
\label{normification exact}
\begin{tikzcd}
0 \arrow[r] & \ker V \arrow[r] & V \arrow[r,"\pi"] & \frac{V}{\ker V} \arrow[r] & 0
\end{tikzcd}
\end{equation}
(so the composite of any two arrows in the diagram (\ref{normification exact}) is the zero map, $\pi$ is the natural projection of $V$ onto $V/\ker V$, and $\ker V$ is the kernel of $\pi$).
We define a norm on $V/\ker V$ by
$$||\pi(v)|| = ||v||.$$
We define the normalization of $V$ to be $V/\ker V$.

To check the universal property, let $T: V \to W$ be a linear map into a normed space $W$ such that $||Tv|| = ||v||$.
In particular, if $v \in \ker V$, $||Tv|| = 0$ so $Tv = 0$ and hence $v \in \ker T$.
Thus we may define $T'(\pi(v)) = Tv$; this is well-defined because if $\pi(v) = \pi(v')$, then $v - v' \in \ker V$ and hence in $\ker T$.
\end{proof}

We conclude this section with a useful consequence of the triangle inequality.
\begin{lemma}[reverse triangle inequality]
For every $v, w$ in a normed space,
$$|||v|| - ||w||| \leq ||v - w||.$$
\end{lemma}
\begin{proof}
One has
$$||v|| = ||v + w - w|| \leq ||v - w|| + ||w||$$
so
$$||v|| - ||w|| \leq ||v - w||.$$
Similarly,
$$||w|| - ||v|| \leq ||v - w||.$$
But $\max(||v|| - ||w||, ||w|| - ||v||) = |||v|| - ||w|||$.
\end{proof}

\section{Banach spaces}
\label{Banach space appendix}
The reader should be familiar with this material before reading Chapter \ref{measureChapter}.

The defining property of $\RR$ is that every Cauchy sequence in $\RR$ converges.
However, this property is not true for normed spaces, as the following example will show.
\begin{example}
Let $C[0, 1]$ denote the vector space of continuous functions $[0, 1] \to \CC$.
We turn $C[0, 1]$ into a normed space by introducing a Euclidean-type norm,
$$||f||_2^2 = \int_0^1 |f(x)|^2 ~dx.$$
It is easy to check that $||f||_2$ is a seminorm, and to see that it is a norm, note that if $\int_0^1 |g| = 0$, then for every $\varepsilon > 0$ there is a partition of $[0, 1]$ into intervals $I_1, \dots, I_n$ such that for every $i$, the length of $I_i$ is $<\varepsilon$ and the midpoint $I_i^*$ of $I_i$ satisfies $|g(I_i^*)| < \varepsilon$.
Thus the set of points $x$ such that $|g(x)| < \varepsilon$ is $\varepsilon$-dense (i.e. for every $y$ there is an $x$ such that $|x - y| < \varepsilon$ and $|g(x)| < \varepsilon$).
Taking $\varepsilon \to 0$ we see that the set of points $x$ such that $g(x) = 0$ is dense (i.e. for every $\delta > 0$ and every $y$ we can find an $x$ with $|x - y| < \delta$ and $g(y) = 0$).
But $g$ is continuous so $g = 0$.

Now define
$$f_n(x) = \begin{cases}
0, &x \leq 1/2 - 1/n\\
n(x - 1/2 - 1/n), &1/2 - 1/n \leq x \leq 1/2 + 1/n\\
2, &x \geq 1/2 + 1/n
\end{cases}.$$
TODO: Draw a picture
To see that $(f_n)_n$ is Cauchy, note that if $m > n$ then
$$||f_n - f_m||_2^2 = \int_{1/2-1/n}^{1/2+1/n} (m(x - 1/2 - 1/m) - n(x - 1/2 - 1/n))^2~dx$$
and the integrand satisfies
$$(m(x - 1/2 - 1/m) - n(x - 1/2 - 1/n))^2 \leq 8.$$
Thus
$$||f_n - f_m||_2^2 \leq \int_{1/2-1/n}^{1/2+1/n} 8~dx = \frac{16}{n}.$$
Therefore
$$||f_n - f_m||_2 \leq \frac{4}{\sqrt n} \to 0$$
as $n \to \infty$. However, it is not too hard to check that if
$$f(x) = \begin{cases}
0, x \leq 1/2\\
1, x > 1/2
\end{cases}$$
then
$$\lim_{n \to \infty} ||f_n - f||_2 = 0;$$
yet $f$ is not continuous, and $||\cdot||_2$ is a norm, so $f_n$ cannot converge to any continuous function.
\end{example}

\begin{definition}
A \dfn{Banach space} is a normed space for which every Cauchy sequence converges.
\end{definition}

\begin{example}
The $p$-norms $||\cdot||_p$ that we defined on $\RR^d$ turn $\RR^d$ into a Banach space.
This is an exercise in the fact that every Cauchy sequence on $\RR$ converges.
\end{example}

The advantage of working in Banach spaces rather than general normed spaces is that it is meaningful to talk about infinite sums in Banach spaces. Indeed, if $(x_n)_n$ is a sequence of elements in a Banach space $X$, we define
\begin{equation}
\label{banach space series}
\sum_{n=1}^\infty x_n = \lim_{N \to \infty} \sum_{n=1}^N x_n
\end{equation}
whenever the limit on the right-hand side of (\ref{banach space series}) makes sense; thus, $y = \sum_n x_n$ if for every $\varepsilon > 0$, there is an $N$ such that
$$\left|\left|y - \sum_{n=1}^N x_n\right|\right| < \varepsilon.$$
In fact, the right-hand side of (\ref{banach space series}) makes sense as long as the partial sums are Cauchy; thus, for every $\varepsilon > 0$, there is an $N$ such that
$$\left|\left|\sum_{n=N}^\infty x_n\right|\right| < \varepsilon.$$
By the triangle inequality, it in fact suffices to show that for every $\varepsilon > 0$ there is an $N$ such that
\begin{equation}
\label{absolute convergence}
\sum_{n=N}^\infty ||x_n|| < \varepsilon
\end{equation}
to show that the $(x_n)_n$ are summable.
\begin{definition}
Let $(x_n)_n$ be a sequence in a Banach space. If for every $\varepsilon > 0$ there is an $N$ such that (\ref{absolute convergence}) holds, we say that $(x_n)_n$ is \dfn{absolutely convergent} or \dfn{absolutely summable}.
\end{definition}

Just as we had a universal way to turn any seminormed space into a normed space, its normalization, we have a universal way to turn any normed space (hence any seminormed space) into a Banach space by adding limits to each of its Cauchy sequences.
\begin{theorem}
\label{completion exists}
For every normed space $V$ there is a Banach space $W$, called the \dfn{completion} of $V$, such that:
\begin{enumerate}
\item There is an injective linear map $\iota: V \to W$ such that for every $v$, $||\iota(v)|| = ||v||$.
\item The image of $\iota$ is dense in $W$.
\item The completion satisfies the \dfn{universal property of the completion}: for any Banach space $X$ and any linear map $T: V \to X$ such that for every $v$, $||Tv|| = ||v||$, then there is a linear map $T': W \to X$ such that the diagram
$$\begin{tikzcd}
V \arrow[rr,"T"] \arrow[dr,"\iota"] && X\\
& W \arrow[ur,"T'"]
\end{tikzcd}$$
commutes.
\end{enumerate}
\end{theorem}
As with the normalization, the completion is unique for category-theoretic reasons, and the reader should omit the proof of this theorem on first reading.
\begin{proof}
To do this, let $V$ be a normed space and let $\Cau (V)$ be the vector space of all Cauchy sequences in $V$.
(We let $x$ denote a Cauchy sequence $(x_n)_n$.)
Then $\Cau (V)$ is a seminormed space, where
$$||x|| = \lim_{n \to \infty} ||x_n||.$$
(To see that the limit exists, note that if $x$ is a Cauchy sequence then $n \mapsto ||x_n||$ is a Cauchy sequence in $\RR$, so it converges.)
Let $W$ be the normalization of $\Cau (V)$ and $\pi: \Cau (V) \to W$ the natural projection.

We claim that $W$ is the completion of $V$.
First, $W$ is a Banach space, because if $x \in \Cau (W)$, then we can choose $\widetilde x \in \Cau (\Cau (V))$ such that for every $n$,
$$\pi(\widetilde x_n) = x_n.$$
Now $\widetilde x_n$ is a Cauchy sequence, say $(\widetilde x_{n,m})_m$. If we replace $\widetilde x_n$ with a subsequence (in $m$) of $\widetilde x_n$, $\pi(\widetilde x_n)$ will not change.
Thus we might as well assume that for any two $m,m'$,
$$||\widetilde x_{n,m} - \widetilde x_{n,m'}|| < \frac{1}{n}.$$
Now let
$$\widetilde y_n = \widetilde x_{n,n}.$$
We want $x_n \to \pi(\widetilde y)$ but it's not obvious that $\widetilde y$ is a Cauchy sequence.

Let $\varepsilon > 0$, so there is an $N > \varepsilon^{-1}$ such that for every $n, n' \geq N$,
$$||x_n - x_{n'}|| < \varepsilon$$
and for any $j$,
\begin{align*}
  ||\widetilde y_n - \widetilde y_{n'}|| &= ||\widetilde x_{n,n} - \widetilde x_{n',n'}||\\
  &\leq ||\widetilde x_{n,n} - \widetilde x_{n,j}|| + ||\widetilde x_{n,j} - \widetilde x_{n',j}|| + ||\widetilde x_{n',j} - \widetilde x_{n',n'}||\\
  &\leq \frac{1}{n} + ||\widetilde x_{n,j} - \widetilde x_{n',j}|| + \frac{1}{n'}\\
  &< 2\varepsilon + ||\widetilde x_{n,j} - \widetilde x_{n',j}||.
\end{align*}
But
$$\lim_{j \to \infty} ||\widetilde x_{n,j} - \widetilde x_{n',j}|| = 0$$
and since $j$ was large enough we can take $||\widetilde x_{n,j} - \widetilde x_{n',j}|| < \varepsilon$. Then
$$||\widetilde y_n - \widetilde y_{n'}|| \lesssim \varepsilon.$$
Therefore $\widetilde y \in \Cau (V)$, and we can define
$$y = \pi(\widetilde y).$$

We similarly choose $N > \varepsilon^{-1}$ such that for every $n, n' \geq N$,
$$||\widetilde y_n - \widetilde y_{n'}|| < \varepsilon.$$
Then
$$||\widetilde x_{n',n} - \widetilde y_n|| \leq ||\widetilde x_{n',n} - \widetilde x_{n',n'}|| + ||\widetilde y_{n'} - \widetilde y_n|| \lesssim \varepsilon.$$
But
$$\lim_{m \to \infty} ||\widetilde x_{n,m} - \widetilde y_m|| = ||x_n - y||$$
so $\lim_n x_n = y$.

We can define
\begin{equation}
\label{inclusion into completion}
\iota(v) \mapsto \pi(v, v, v, \dots).
\end{equation}
Clearly the limit of the Cauchy sequence on the right-hand side of (\ref{inclusion into completion}) is $v$, so $||\iota(v)|| = ||v||$.

To see that $\iota(V)$ is dense in $W$, let $x \in W$ and choose $\widetilde x \in \Cau (V)$ with $\pi(\widetilde x) = x$.
Now $\widetilde x$ is a Cauchy sequence in $V$, so let
$$y_n = \iota(\widetilde x_n);$$
then $y \in \Cau (W)$, and the reader can check that $\lim_n y_n = x$.

To check the universal property, let $T: V \to X$ be a linear map of $V$ into a Banach space $X$ such that $||Tv|| = ||v||$.
Thus $T$ is a bounded linear map (c.f. Definition \ref{bounded linear map}). First define $T'(\iota(v)) = v$ for every $v$, thus $T'$ is defined on the dense subspace $V'$ of $W$.
But then $T'$ is continuous, so it extends uniquely to a linear map by Lemma \ref{linear extension}. We leave it to the reader to check $||T'w|| = ||w||$.
\end{proof}

\section{Linear maps}
Though we will use this material before then, the reader need not familiarize themself with this material until Chapter TODO.
\begin{definition}
\label{bounded linear map}
Let $T: V\to W$ be a linear map between normed spaces. We say that $T$ is a \dfn{bounded linear map} if there is a $C>0$ such that for every $v \in V$,
$$||Tv|| \leq C||v||.$$
The infima of all choices of $C$ is called the \dfn{operator norm} of $T$, denoted $||T||$.
We denote by $B(V \to W)$ the space of bounded linear maps.
\end{definition}

\begin{lemma}
Let $V,W$ be normed spaces.
Then the operator norm is a norm on $B(V \to W)$.
\end{lemma}
We leave the routine proof to the reader.

\begin{lemma}
\label{linear extension}
Let $T: V \to W$ be a bounded linear map between normed spaces, and suppose that $V$ is a dense subspace of a normed space $X$.
Then there is a unique extension of $T$ to a bounded linear map $X \to W$, which has the same operator norm.
\end{lemma}
\begin{proof}
Let $x \in X$, and let $(x_n)_n$ be a sequence in $V$ with $\lim_n x_n = x$.
Define $Tx = \lim_n Tx_n$.
We let $||T||_V$ denote the operator norm of $T$ with domain $V$ and $||T||_X$ with domain $X$.

To see that this is well-defined, suppose that $\lim_n x_n' = x$ as well. Then
$$\lim_{n \to \infty} ||Tx_n - Tx_n'|| \leq \lim_{n \to \infty} ||T||_V \cdot||x_n - x_n'|| = 0$$
since $x_n \to x$ and $x_n' \to x$.
Thus $Tx$ does not depend on the choice of Cauchy sequence which approximates $x$.
Similarly,
$$||Tx|| = \lim_{n \to \infty} ||Tx_n|| \leq \lim_{n \to \infty} ||T||_V \cdot||x_n|| = ||T||_V ||x_n||.$$
Thus $||T||_X \leq ||T||_V$, but $V$ is a subspace of $X$ so clearly $||T||_V \leq ||T||_X$.

To check that $T$ is linear on $X$, note that $T$ is continuous and hence can be commuted with limits. This means that
$$T(x + y) = \lim_{n \to \infty} T(x_n + y_n) = \lim_{n \to \infty} Tx_n + Ty_n = Tx + Ty$$
where $(x_n)_n$ and $(y_n)_n$ are appropriately chosen Cauchy sequences.
The proof that $T(cx) = cTx$ is similar.
\end{proof}

\section{Properties of Banach spaces}
\begin{definition}
Let $B$ be a Banach space and let $X \subseteq B$. We say that $X$ is \dfn{separable} if there is a countable dense subset of $X$.
\end{definition}

If $X$ is separable and $Y \subseteq X$, then $Y$ is separable; in fact, if $C$ is countable and dense in $X$, then $C \cap Y$ is countable and dense in $Y$.

\begin{example}
$\CC^n$ is separable, since $\{(\alpha_1 + i\beta_1, \dots, \alpha_n + i \beta_n): \alpha_i, \beta_i \in \QQ\}$ is countable and dense in $\CC^n$.
In particular any finite-dimensional space is separable. Moreover, most spaces that we consider will turn out to be separable.
\end{example}

\begin{example}
\label{nonseparable space}
Let $X$ be the space of all bounded functions $\CC \to \CC$, where
$$||f - g|| = \sup_{z \in \CC}|f(z) - g(z)|.$$
Then $X$ is not separable. In fact, if $z \in \CC$, let $1_z(z) = 1$ and $1_z(w) = 0$ if $w \neq z$.
Then $Y=\{1_z: z \in \CC\}$ is an uncountable subset of $X$ such that for every $z_1 \neq z_2$, $||1_{z_1} - 1_{z_2}|| =1$.
So $Y$ is discrete and uncountable, and hence cannot be separable.
\end{example}

\section{Convexity}



\chapter{Foundations}
In this appendix we treat the foundations of math: set theory, category thoery, and point-set topology.
We omit the proofs, not because they are uninteresting or unworthy of being learned, but because these topics are not analysis; this is an analysis book, and it is reasonable to use material that is not analysis as a black box.

\section{Axiomatic set theory}
We assume familiarity with naive set theory; this will suffice for everything in this text except for a few examples, which require more sophisticated set-theoretic techniques that we record here.
We refer the reader to CITE:Kunen for more details.

Let us record the first few axioms of set theory.
It will be convenient to assume that \emph{every} mathematical object is a set.
This is no loss of generality, because one can define the natural numbers by declaring that $0 = \emptyset$ and for every natural number $n$, $n = \{0, \dots, n - 1\}$.
The definition of real numbers in terms of Dedekind cuts, where each real number $x$ is by definition the set $\{y \in \QQ: y < x\}$, is similar.
More generally, any mathematical object can be encoded as a set in some appropriate way.

The first three axioms are presumably uncontroversial to anyone who has studied naive set theory.

\begin{axiom}[extensionality]
For all sets $x, y, z$, if $z \in x$ implies $z \in y$, and $z \in y$ implies $z \in x$, then $x = y$.
\end{axiom}
\begin{axiom}[pairing]
For all sets $x, y$, there exists a set $z$, usually denoted $\{x, y\}$, such that for every set $w$, $w \in z$ iff $w = x$ or $w = y$.
\end{axiom}
\begin{axiom}[union]
For all sets $x, y$, there exists a set $z$, usually denoted $x \cup y$, such that for every set $w$, $w \in z$ iff $w \in x$ or $w \in y$.
\end{axiom}

Now to avoid Russell's paradox, and avoid other logic technicalities we want to forbid that a set include itself.
\begin{axiom}[foundation]
For all sets $x$ such that there is a set $w \in x$, there is a set $y \in x$ such that for all sets $z \in y$, $z \notin x$.
\end{axiom}
If $x \in x$, then by pairing, $\{x\}$ is a set, and this contradicts foundation.

A \dfn{first-order formula} with $N$ free variables in the language of set theory is a string consisting only of the symbols $\exists,\forall,\to,\neg,\in,(,)$, and variables $x_1,x_2,\cdots, y_1,\dots,y_N$ that refer to sets, which is meaningful (so the formula $)\forall \implies()$ is not a first-order formula), such that the $x_i$ always appear after a quantifier $\forall$ or $\exists$, and the $y_i$ never do.
Plugging in sets for the $y_i$ gives a statement which is either true or false.
For example, $\forall x_1((x_1 \in y_1) \to (x_1 \in y_2))$ is a formula which asserts that $y_1 \subseteq y_2$.

First-order formulae allow us to assert the existence of subsets.
\begin{axiom}[restricted comprehension schema]
For all sets $x$, first order-formulae $\varphi$ with $N+1$ free variables, and sets $w_1, \dots, w_N$, there is a set $y$ such that $z \in y$ iff $z \in x$ and $\varphi(z, w_1, \dots, w_N)$ is true.
\end{axiom}
Henceforth we adopt the usual notation $\subseteq$, $\subset$, $\{y \in x: \varphi(y)\}$, etc.
We can also now define the empty set $\emptyset$ and various other interesting sets.

Already we have developed enough to study finite sets.
For example, we can define intersection $\cap$ and ordered pairs $(x, y)$ in terms of the notions we introduced in the above axioms.
We can also define functions; a function $f: X \to Y$ is just a set $f$ of ordered pairs $(x, y)$ such that for every $x \in X$ there is exactly one $y \in Y$ such that $(x, y) \in f$ (though we usually write $f(x) = y$ to mean $(x, y) \in f$).
So we can talk about injections and bijections, and ask if two sets have the same cardinality.
\begin{definition}
Let $x, y$ be sets. We say that $x, y$ have the same \dfn{cardinality}, and write $x \cong y$, if there is a bijection $x \to y$.
\end{definition}
\begin{theorem}[Cantor-Bernstein]
Let $x, y$ be sets. Then $x \cong y$ iff there are injections $x \to y$ and $y \to x$.
\end{theorem}
In particular, we can ask if a set is finite, countable, etc.
\begin{definition}
A \dfn{finite set} is a set $x$ such that for every $y \subseteq x$, if $x \cong y$, then $x = y$.
Otherwise, the set $x$ is an \dfn{infinite set}.
\end{definition}
However, it is not obvious (and in fact, may even be false) that there is an infinite set at this stage.
\begin{axiom}[infinity]
There exists a set $x$, usually denoted $\NN$, such that $\emptyset \in x$ and for all sets $y \in x$, $\{y \cup \{y\}\} \in \NN$.
\end{axiom}
From $\NN$ we may construct $\ZZ$, $\QQ$, and other familiar countable objects.
\begin{definition}
A set $x$ is a \dfn{countable set} if there is an injection $x \to \NN$.
Otherwise, the set $x$ is an \dfn{uncountable set}.
\end{definition}
Again we get stuck; it is not obvious (and may be false) that there can be uncountable sets.

The axioms that follow are highly dangerous. They assert the existence of deeply infinitary sets, whose elements cannot be easily described.
This is the essence of the paradoxical examples that we will use axiomatic set theory to prove the existence of.
Famously, Lebesgue rejected the axiom of choice (while implicitly using weak forms of it); modern constructivists, finitists, and intuitionists reject some or all of the following axioms (or even the axiom of infinity).
\begin{axiom}[power set]
For every set $x$, there exists a set $y$, usually denoted $2^x$, such that for all sets $z$, $z \in y$ iff $z \subseteq x$.
\end{axiom}
The axioms that we have developed up to this point define what is known as \dfn{Zermelo set theory}.

There is clearly an injection $x \to 2^x$, given by $y \mapsto \{y\}$.
\begin{theorem}[Cantor's diagonal argument]
For every set $x$, there is no injection $2^x \to x$.
\end{theorem}
In particular, $2^\NN$ is uncountable.
We may now prove the existence of $\RR$, $2^\RR$, the topology of $\RR$, and so on. For example, we have:
\begin{theorem}[Dedekind]
There exists a set $\RR$, whose elements are nonempty proper subsets $x$ of $\QQ$ such that for each $q \in x$, if $r \in \QQ$ and $r < q$ then $r \in x$, and there is an $s \in x$ such that $s > x$.
There exists a ring structure on $\RR$ which turns $\RR$ into an ordered field of characteristic $0$ such that for every set $X \in 2^\RR$, $\sup X$ is well-defined.
Moreover, $\RR$ is unique up to unique isomorphism of ordered fields.
\end{theorem}
We now pause to introduce the notion of \dfn{transfinite induction}, which we will use a few times.
\begin{definition}
A \dfn{transitive set} is a set $x$ such that $x \subset 2^x$.
\end{definition}
Thus every element of a transitive set is also a subset of $x$.
\begin{definition}
\label{ordinal dfn}
An \dfn{ordinal} is a transitive set $x$ such that for every $y \in x$, $y$ is an ordinal.
\end{definition}
This definition may seem circular -- but it is not. $\emptyset$ is an ordinal, usually denoted $0$ when we think of it as an ordinal, and every natural number is an ordinal (why?), but so is $\NN$, which we usually denote $\omega$ when we think of it as an ordinal.
From there we keep going: given an ordinal $\alpha$ we define its \dfn{successor} $\alpha + 1 = \alpha \cup \{\alpha\}$.
Not every ordinal is a successor or $0$; for example $\omega$ was not. Such ordinals are known as \dfn{limits}.
They are limit points in the order topology.
\begin{theorem}[transfinite induction]
\label{transfinite induction}
Fix a limit ordinal $\kappa$. Let $X$ be a set of ordinals such that:
\begin{enumerate}
\item $0 \in X$.
\item For every ordinal $\alpha \in X$, $\alpha + 1 \in X$.
\item If $\delta < \kappa$ is a limit ordinal and for every $\alpha < \delta$, $\alpha \in X$, then $\delta \in X$.
\end{enumerate}
Then $X = \kappa$.
\end{theorem}
The trivial example of transfinite induction is when $\kappa = \omega$; then there are no limit ordinals to consider and the theorem collapses down to induction on $\NN$.
From this we can introduce \dfn{transfinite recursion}.
Suppose that $\kappa$ is a limit ordinal, and we want to define sets $x_\alpha$ for every $\alpha < \kappa$. Then we just have to:
\begin{enumerate}
\item Define $x_0$.
\item Show that if we can define $x_\alpha$, then we can define $x_{\alpha + 1}$.
\item Show that for every limit ordinal $\delta < \kappa$, if we can for every $\alpha < \delta$ define $x_\alpha$, then we can define $x_\delta$.
\end{enumerate}
Then the theorem of transfinite induction will imply that, for every $\alpha < \kappa$, $x_\alpha$ is defined.

The set of all countable ordinals is an uncountable ordinal (in fact, the smallest uncountable ordinal), which we denote $\omega_1$.

If $X$ is a set of ordinals, then $X$ has a least element; this is similar to the fact that every set of natural numbers has a least element.
\begin{definition}
An ordinal $\kappa$ is a \dfn{cardinal} if $\kappa$ is the least element of $\{\alpha: \alpha \cong \kappa\}$.
\end{definition}
The idea is that cardinals should be canonical representatives of the equivalence class of all sets with a given cardinality.

We let $\aleph_0$ be the least infinite cardinal (thus $\aleph_0 = \omega$), and for every cardinal $\aleph_n$, we define $\aleph_{n+1}$ to be the least cardinal greater than $\aleph_n$.
We would like to continue this recursion and let $\aleph_\omega$ be the least cardinal greater than $\aleph_n$ for every $n \in \NN$.
It is not possible to define $\aleph_\omega$ yet, but we can do so with the help of a new axiom schema.

\begin{axiom}[replacement schema]
For every set $X$, every first-order formula with $N+2$ free variables $\varphi$, and all sets $w_1, \dots, w_N$, if for every $x \in X$ there is exactly one set $y$ such that $\varphi(x, y, w_1, \dots, w_N)$ is true, then there is a set $z$ such that $y \in z$ iff there is an $x \in X$ such that $\varphi(x, y, w_1, \dots, w_N)$ is true.
\end{axiom}
In other words, if $F$ is a function which can be explicitly defined by a first-order formula in terms of $N$ parameters $w_1, \dots, w_N$, then the image of $F$ is a set.
It follows from the replacement schema that $\aleph_\alpha$ is defined for every ordinal $\alpha$.

Zermelo set theory along with the replacement schema is known as \dfn{Zermelo-Fraenkel set theory}.

Zermelo-Fraenkel set theory cannot prove that every set is in bijection with a cardinal, or that every vector space has a basis.
To prove that every set is in bijection with a cardinal, Zermelo introduced the so-called axiom of choice.

\begin{axiom}[choice]
\label{axiom of choice}
For every surjection $f: X \to Y$ there is an injection $g: Y \to X$ such that for every $y \in Y$, $f(g(y)) = y$.
\end{axiom}
In other words if $f: X \to Y$ is a surjection then for every $y \in Y$ we may choose $x \in X$ so that $f(x) = y$.
This follows from Zermelo-Fraenkel set theory if we only have to choose finitely many $x$, but in general, $X$ may be uncountable, so that we may have to make uncountably many choices when we define $g$.

The above axioms comprise \dfn{Zermelo-Fraenkel set theory with choice}.
Throughout the text, we assume Zermelo-Fraenkel set theory with choice.

\begin{theorem}[Zermelo's well-ordering theorem]
\label{well-ordering theorem}
For every set $x$ there is a unique cardinal $\kappa$ such that $x \cong \kappa$.
\end{theorem}

\begin{definition}
For every set $x$, the unique cardinal $\kappa$ such that $x \cong \kappa$ is called the \dfn{cardinality} of $x$.
We denote the cardinality of $x$ by $\card x$.
\end{definition}

\begin{proof}[Proof of Zermelo's well-ordering theorem]
Uniqueness is obvious since $\cong$ is an equivalence relation.
So we need a cardinal $\kappa$ and a bijection $f: \kappa \to x$.

Let $\lambda$ be a cardinal. We first build a function $f_\lambda: \lambda \to x$ by transfinite recursion, using the axiom of choice.
Choose $y \in x$ and set $f_\lambda(0) = y$.
Suppose $\alpha < \lambda$ and we have defined $f_\lambda(\beta)$ for $\beta < \alpha$.
Choose $z \in x$ such that for every such $\beta$, $f_\lambda(\beta) \neq z$, if such a $z$ exists; otherwise set $f_\lambda(\beta) = y$.
By transfinite recursion, this defines a function $f_\lambda$. If $f_\lambda$ is a bijection we can set $\kappa = \lambda$ and we're done.

Otherwise suppose $f_\lambda$ is not injective. Then there is an ordinal $\beta < \lambda$ (so in particular, $\card \beta < \lambda$) such that $f_\lambda|\beta$ is injective and $\beta$ is maximal possible (so in particular, $f_\lambda|\beta$ is surjective), by construction of $f_\lambda$.
Let $\kappa = \card \beta$ and choose $g: \beta \to \kappa$.
Then set $f = (f_\lambda|\beta) \circ g^{-1}$.

Finally suppose $f_\lambda$ is not surjective. Let $\delta = \card 2^\lambda$ and consider $f_\delta$ instead.
If the process of passing from $\lambda$ to $\delta$ does not halt, then there are arbitrarily large cardinals $\delta$ which admit injections $\delta \to x$, so the universe of all sets embeds in $x$.
This breaks Russell's paradox.
\end{proof}

Note that while we technically don't need Zermelo's well-ordering theorem to prove the Cantor-Bernstein theorem, it is an immediate consequence.

\begin{definition}
\label{beth dfn}
We define $\beth_0 = \aleph_0$, and for every $\alpha$, $\beth_{\alpha + 1}$ to be the cardinality of $2^{\beth_\alpha}$.
If $\delta$ is a limit ordinal we let $\beth_\delta$ be the cardinality of $\bigcup_{\alpha < \delta} \beth_\alpha$.
\end{definition}

By Cantor's diagonal argument, $\beth_\alpha < \beth_\beta$ whenever $\alpha < \beta$.

\begin{theorem}
\label{cardinal arithmetic trivial}
Let $\lambda \leq \kappa$ be infinite cardinals.
Suppose that $X$ is a set of cardinality $\lambda$, whose elements are sets of cardinality $\kappa$. Then $\bigcup_{x \in X} x$ has cardinality $\kappa$.
Moreover, if $\lambda < \kappa$ and $\kappa = \beth_\alpha$ for some $\alpha$, then the Cartesian product of $\lambda$ many copies of $\kappa$ has cardinality $\kappa$.
\end{theorem}

In particular, if $\lambda$ is countable or $\lambda = \beth_1$, then a union of $\lambda$ many sets of cardinality $\beth_1$ has cardinality $\beth_1$.

\begin{theorem}
\label{cardinality of topology}
Let $\mathcal T$ be the set of all open subsets of $\RR$; then $\mathcal T$ has cardinality $\beth_1$.
\end{theorem}

Another consequence of Zermelo's well-ordering theorem is Zorn's lemma, which is frequently useful.
The point is to allow us to build up ``deeply infinitary" objects (that cannot be completed in $\omega$ many steps!) without explicitly using transfinite recursion.
Of course the proof will be by transfinite recursion, but mathematicians who are not logicians tend to find the transfinite terrifying, and use Zorn's lemma as something of a black box.

\begin{definition}
Let $\PP$ be a partially ordered set. A \dfn{chain} $\mathcal C$ in $\PP$ is a set such that for every $x, y \in \mathcal C$, $x \leq y$ or $y \leq x$.
\end{definition}

\begin{theorem}[Zorn's lemma]
Let $\PP$ be a partially ordered set such that every chain in $\PP$ has an upper bound.
Then $\PP$ has a maximal element.
\end{theorem}
\begin{proof}
Let $\kappa = \card \PP$ and choose a bijection $f: \kappa \to \PP$.
We will define an ordinal $\delta$ and an injection $g: \delta + 1 \to \PP$ as follows.
Let $g(0) = f(0)$. If $g(\alpha)$ has been defined and is not maximal, there is a $\beta < \kappa$ such that $f(\beta) > g(\alpha)$; choose the least such $\beta$, and set $g(\alpha + 1) = f(\beta)$.
If $g(\alpha)$ has been defined for all $\alpha < \gamma$, then let $\mathcal C_\gamma = \{g(\alpha): \alpha < \gamma\}$; then $\mathcal C_\gamma$ is a chain, so it has an upper bound, which we define to be $g(\gamma)$.

If the above process stops at some $\delta$, then $g(\delta)$ is maximal in $\PP$.
Otherwise, there are arbitrarily large ordinals $\delta$ such that there are injections $\delta \to \PP$, hence $\delta \to \kappa$.
In particular we could take $\delta = \card 2^\kappa$ and get an injection $2^\kappa \to \kappa$, a contradiction.
\end{proof}

\section{Universal properties}
This section has nothing to do with measure theory.
It solely exists to justify certain algebraic handwaves in the appendices.
The reader who is interested in category theory should read TODO:Cite Aluffi and Riehl, and most other readers, except those who are completely ignorant of algebra, should ignore this section entirely.

To avoid technicalities we adjoin a new axiom to the Zermelo-Fraenkel set theory with the axiom of choice to obtain a weak form of \dfn{Tarski-Grothendieck set theory}.

\begin{definition}
A transitive set $U$ is a \dfn{universe} if $\NN \in U$, $U$ is closed under pairing and power set, and for every subset $V \subset U$ of strictly less cardinality than $U$, $\bigcup_{x \in V} x \in U$.
\end{definition}
\begin{axiom}[universe]
There exists a universe.
\end{axiom}
Henceforth we fix a universe $U$.
One can show that universes are closed under every relevant operation that mathematicians might care about, so any object that will ever appear in this book is contained in $U$.
For example $\RR \in U$, any measurable subset of $\RR$ is in $U$, and any Banach space we ever consider is in $U$.
\begin{definition}
A \dfn{small set} is an element of $U$.
\end{definition}
So anything that we will ever have reason to care about, except in this section, is small.

\begin{definition}
A \dfn{category} $C$ is a set, whose elements are called \dfn{objects}, along with small sets $\Hom(x, y)$ for all objects $x, y \in C$, whose elements are called \dfn{morphisms} from $x$ to $y$, equipped with \dfn{composition} operations
\begin{align*}
\Hom(x, y) \times \Hom(y, z) &\to \Hom(x, z)\\
(\varphi, \psi) &\mapsto \psi \varphi
\end{align*}
such that:
\begin{enumerate}
\item For every object $x \in C$, $\Hom(x, x)$ contains a morphism $1_x$ which is an \dfn{identity} in the sense that for all $\varphi \in \Hom(x, y)$, $\varphi = \varphi 1_x$ and for every $\psi \in \Hom(y, x)$, $\psi = 1_x \psi$.
\item Composition is \dfn{associative} in the sense that $\psi(\varphi\rho) = (\psi\varphi)\rho$ whenever $\psi\varphi$ and $\varphi\rho$ are defined.
\end{enumerate}
\end{definition}

For example $U$ forms a category, whose objects are small sets, and whose morphisms are defined by letting $\Hom(x, y)$ be the set of all functions $x \to y$.
We dare not define a category whose objects consist of all sets, due to Russell's paradox; but we will abuse terminology and call $U$ the category of sets all the same. We denote it by $\Set$.
Similarly we define $\Grp$, the category of (small) groups where the morphisms are group homomorphisms, and $\Vect(K)$, the category of (small) vector spaces over a (small) field $K$ where the morphisms are linear maps.

It can be convenient to draw diagrams of morphisms, which are graphs where the nodes are objects, an edge from $x$ to $y$ is a morphism in $\Hom(x, y)$, and the diagram \dfn{commutes} if for any two objects $x, y$ in the diagram and any two paths $\varphi_1\cdots\varphi_n$ and $\psi_1\cdots\psi_m$ from $x, y$, $\varphi_1\cdots\varphi_n = \psi_1\cdots\psi_m$.
For example, the diagram
$$\begin{tikzcd}
x_1 \arrow[r,"\varphi_1"] \arrow[d,"\psi_1"] & x_2 \arrow[d,"\psi_2"]\\
y_1 \arrow[r,"\varphi_2"] & y_2
\end{tikzcd}$$
commutes iff $\psi_1\varphi_2 = \psi_1\varphi_2$.

\begin{definition}
Let $C$ be a category, $x,y \in C$ objects, and $\varphi \in \Hom(x, y)$.
We say that $\varphi$ is an \dfn{isomorphism}, and $x,y$ are \dfn{isomorphic}, if there is a $\psi \in \Hom(y, x)$ which is an \dfn{inverse} to $\varphi$ in the sense that $\varphi\psi = 1_y$ and $\psi\varphi = 1_x$.
\end{definition}

\begin{definition}
Let $C$ be a category and $x \in C$ an object.
We say that $x$ is \dfn{initial} in $C$ if for every $y \in C$ there is a unique morphism in $\Hom(x, y)$.
Similarly we say that $x$ is \dfn{final} if for every $y$ there is a unique morphism in $\Hom(y, x)$.
Either way, we say that $x$ is \dfn{terminal}.
\end{definition}

If $x$ and $y$ are terminal, then there is a \emph{unique} isomorphism between $x$ and $y$.
In category theory one cannot distinguish between two objects between which there is a unique isomorphism, so we abuse terminology and say that $x = y$, even if $x,y$ are not equal in the sense of the axiom of extensionality.
Thus a terminal object, if it exists, is unique.
For example, the trivial vector space is the unique terminal object in $\Vect$.

\begin{definition}
Let $P$ be a property held by objects $x$ in a certain category $C$.
We say that $P$ is a \dfn{universal property} if, for every $x$ that holds property $P$, $x$ is terminal in $C$.
\end{definition}

Therefore an object with a universal property is unique.

We can now make rigorous our handwaving about the universal property of the completion.
Let $V$ be a normed space. We define a category $C(V)$, whose objects are norm-preserving linear maps $V \to X$ into a Banach space $X$, and whose morphisms $\varphi$ are commutative diagrams of linear maps
$$\begin{tikzcd}&X \arrow[dd,"\varphi"]\\
V \arrow[ur] \arrow[dr]\\
&Y.\end{tikzcd}$$
The initial object of $V$ is the inclusion map $V \to W$, where $W$ is the completion of $V$.
Therefore the completion is well-defined if it exists (which it does, by Theorem \ref{completion exists}).
Thus we can take the universal property as the definition of the completion, and take Theorem \ref{completion exists} as just an indication that this definition makes sense; we can then forget about such hideous objects as $\Cau(\Cau(V))$.

We paraphrase the above universal property by saying that the completion of $V$ is the initial Banach space which contains a copy of $V$.
A similar argument shows that $\RR$ is uniquely defined as the initial complete metric space which contains a copy of $\QQ$.

\section{Point-set topology}
We briefly sketch ideas from point-set topology that we will need.
We refer the reader to TODO:Cite Munkres or Bradley for reference.

\begin{definition}
A \dfn{topology} $\mathcal T$ in a set $X$ is a set of subsets of $X$ such that:
\begin{enumerate}
\item $\emptyset, X \in \mathcal T$.
\item If $\mathcal U \subseteq \mathcal T$, then the union $\bigcup_{U \in \mathcal U} U$ of all elements of $\mathcal U$ is also in $\mathcal T$.
\item If $U_1, \dots, U_n \mathcal T$ then $U_1 \cap \cdots \cap U_n \in \mathcal T$.
\end{enumerate}
A pair $(X, \mathcal T)$ is called a \dfn{topological space} and usually just denoted $X$.
Elements of $\mathcal T$ are called \dfn{open sets}.
The complement of an open set is called a \dfn{closed set}.
If $x \in X$, a \dfn{neighborhood} of $x$ is an open set containing $x$.
\end{definition}

It follows that the arbitrary union and finite intersection of open sets is open, and the arbitrary intersection and finite union of closed sets.
The point of a topology is that if $K$ is a closed set, $K$ is closed under taking limits, as we will see.

The obvious examples of topologies on a set $X$ are the discrete topology (wherein every set is open) and the indiscrete topology (wherein the only open sets are $\emptyset, X$).

\begin{definition}
A set $\mathcal B$ of subsets of a set $X$ is called a \dfn{basis} if for all $B_1, \dots, B_n \in \mathcal B$, $B_1 \cap \cdots \cap B_n \in \mathcal B$.
\end{definition}

Every basis generates a topology whose elements are arbitrary unions of sets in the basis.

\begin{definition}
A \dfn{semimetric} $d$ on a set $X$ is a function $d: X \times X \to [0, \infty)$ such that:
\begin{enumerate}
\item For all $x$, $d(x, x) = 0$.
\item For all $x,y$, $d(x, y) = d(y, x)$.
\item The \dfn{triangle inequality}: For all $x,y,z$, $d(x, y) \leq d(x, z) + d(z, x)$.
\end{enumerate}
If the only pairs $(x, y)$ such that $d(x, y) = 0$ are those with $x=y$, we say that $d$ is a \dfn{metric} and call $X = (X,d)$ a \dfn{metric space}.
We call sets of the form $B(x, \varepsilon) = \{y \in X: d(x, y) < \varepsilon\}$, where $\varepsilon > 0$ and $x \in X$, \dfn{open balls}.
\end{definition}

The open balls of a semimetric on $X$ form a basis of subsets of $X$, so every semimetric induces a topology, whose open sets are unions of open balls.

\begin{example}
Every seminorm induces a semimetric by $d(x, y) = ||x - y||$.
In particular, $\RR^d$ has a norm (namely its absolute value), so $\RR^d$ is a topological space.
\end{example}

We now describe two different constructions of spaces: product spaces and subspaces.

\begin{definition}
Let $\mathcal X$ be a set of topological spaces. The \dfn{product space} $\prod \mathcal X$ is the space $X$ which has $\prod \mathcal X$ as a set, and has the smallest topology $\mathcal T$ such that for every open set $U \subseteq Y$, $Y \in \mathcal X$, $\pi: X \to Y$ the canonical projection, $\pi^{-1}(U) \in \mathcal T$.
\end{definition}

For example $X \times Y$ is a product space. The open sets are generated by those of the form $U \times V$ where $U \subseteq X$ and $V \subseteq Y$ are open.

\begin{definition}
Let $X$ be a topological space and $Y \subseteq X$ a subset.
We say that $U \subseteq Y$ is \dfn{open} in $Y$ (or \dfn{relatively open}) if there is an open set $V \subseteq X$ such that $U = V \cap Y$.
With respect to that topology, we call $Y$ a \dfn{subspace} of $X$.
\end{definition}

So a subspace is a topological space in its own right.

Topological spaces allow us to define continuity in a high level of abstraction. This agrees with the usual definition of a continuous function $[0, 1] \to \RR$.

\begin{definition}
Let $X,Y$ be topological spaces and $f: X \to Y$. We say that $f$ is \dfn{continuous} if for every $U \subseteq Y$ open, $f^{-1}(Y)$ is open.
\end{definition}

\begin{lemma}
Let $X,Y$ be metric spaces and $f: X \to Y$. Then $f$ is continuous iff for every $\varepsilon > 0$ and $x \in X$, there is a $\delta > 0$ such that for every $x' \in X$, if $d(x, x') < \delta$, then $d(f(x), f(x')) < \varepsilon$.
\end{lemma}

\begin{lemma}
Let $\mathcal X$ be a set of topological spaces. Then $X = \prod \mathcal X$ satisfies the \dfn{universal property of products} in the category of topological spaces: for every $Y \in \mathcal X$, $\pi_Y: X \to Y$ the canonical projection, $\pi_Y$ is continuous, and for every topological space $Z$, if we are given maps $f_Y: Z \to Y$ for every $Y$, there is a unique map $f: Z \to X$ such that $\pi_Y \circ f = f_Y$.
\end{lemma}

We now want to talk about limits. Unfortunately at the level of abstraction we are working at, this proves quite tricky.

\begin{definition}
A \dfn{directed set} is a partially ordered set $\PP$ such that for all $\alpha_1, \dots, \alpha_k \in \PP$, $\sup(\alpha_1, \dots, \alpha_k)$ exists.
\end{definition}

For example $\NN$ is directed, and in fact so is any ordinal. A topology, equipped with the relation $\subseteq$, is also directed.

\begin{definition}
Let $\PP$ be a directed set and $X$ a topological space.
A \dfn{net} in $X$, indexed by $\PP$, is a function $\PP \to X$. If $x$ is a net, we usually write $x_\alpha$ instead of $x(\alpha)$.
A \dfn{sequence} is a net indexed by $\NN$.
\end{definition}

\begin{definition}
Let $\PP$ be a directed set and $X$ a topological space. Let $x$ be a net in $X$ indexed by $\PP$. Let $x^* \in X$.
We say that $x^*$ is the \dfn{limit} of $x$ if for every neighborhood $U$ of $x$, there is an $\alpha \in \PP$ such that for every $\beta \geq \alpha$, $x_\beta \in U$.
In this case we write $\lim_\gamma x_\gamma = x^*$ or $x_\gamma \to x^*$ if it is clear from context that $\gamma$ is a dummy variable.
If $x$ has a limit we say that $X$ \dfn{converges}.
\end{definition}

For example, in a metric space, $x_n \to x$ iff for every $\varepsilon > 0$ there is a $N \in \NN$ such that for all $n > N$, $d(x_n, x) < \varepsilon$.

\begin{lemma}
Let $f: X \to Y$. Then $f$ is continuous iff for every net $x_\gamma \to x^*$ in $X$, $f(x_\gamma) \to f(x^*)$ in $Y$.
\end{lemma}

\begin{definition}
We say that a function $h: S \to \PP$ is \dfn{cofinal} if for every $\alpha \in \PP$ there is a $\beta$ in the image of $h$ such that $\beta \geq \alpha$.
We say that a function $f: \PP \to \PP$ between directed sets is \dfn{monotone} if for every $\alpha \leq \beta$ in $\PP$, $f(\alpha) \leq f(\beta)$.
\end{definition}

\begin{definition}
Let $X$ be a topological space.
Let $x$ be a net indexed by $\PP$ and $x'$ a net indexed by $\PP'$.
We say that $x'$ is a \dfn{subnet} of $x$ if there is a monotone cofinal function $f: \PP \to \PP'$ such that $x'_\alpha = x_{f(\alpha)}$.
A \dfn{subsequence} is a sequence which is a subnet of a subsequence.
\end{definition}

\begin{lemma}
Let $X$ be a topological space.
A set $K \subseteq X$ is closed iff for every net $x$ in $K$ which converges in $X$, $x$ converges in $K$.
Moreover, given a set $A \subseteq X$, the smallest closed set $K$ containing $A$ is the set of all limits of nets in $A$.
\end{lemma}

Subsequences are significantly easier to work with than subnets, as one can eliminate the need to worry about cofinality; $y$ is a subsequence of $x$ iff there is an increasing sequence $n$ of natural numbers such that $x_{n_k} = y_k$ for all $k \in \NN$.

At this point our definition of limit has three problems:
\begin{enumerate}
\item We want to eliminate the need to choose $\PP$ whenever possible, and just work with $\PP = \NN$.
\item We want to be able to guarantee that if a net has a limit $x^*$, then $x^*$ is unique.
\item We want to be able to guarantee that every net has a subnet which converges.
\end{enumerate}

These are the essences of the definitions of a first-countable, Hausdorff, and compact space respectively.

\begin{definition}
Let $X$ be a topological space and $x \in X$.
A \dfn{fundamental system of neighborhoods} at $x$ is a set $\mathcal U$ of open sets $U \ni x$ such that for every open set $V \ni x$ there is a $U \in \mathcal U$ with $U \subseteq V$.
If, for every $x \in X$, $x$ has a countable fundamental system of neighborhoods, then we say that $X$ is a \dfn{first-countable space}.
\end{definition}

\begin{lemma}
Let $X$ be a first-countable space.
A set $K \subseteq X$ is closed iff for every sequence $x$ in $K$ which converges in $X$, $x$ converges in $K$.
Moreover, given a set $A \subseteq X$, the smallest closed set $K$ containing $A$ is the set of all limits of sequences in $A$.
\end{lemma}

\begin{lemma}
Every metric space is first-countable, and a product of countably many first-countable spaces is first-countable.
\end{lemma}

Now we discuss Hausdorffness.

\begin{lemma}
Let $X$ be a topological space. The following are equivalent:
\begin{enumerate}
\item For every two points $x_1, x_2 \in X$, there are open sets $U_1 \ni x_1$, $U_2 \ni x_2$ such that $U_1 \cap U_2$ is empty.
\item The \dfn{diagonal} $\Delta(X) = \{(x, x): x \in X\}$ is a closed subset of $X \times X$.
\item For every sequence $x \in X$, $x$ has at most one limit.
\end{enumerate}
Here $X \times X$ is viewed as a product space.
\end{lemma}

This lemma motivates a definition.

\begin{definition}
A \dfn{Hausdorff space} is a topological space such that for every two points $x_1, x_2 \in X$, either $x_1 = x_2$ or there are open sets $U_1 \ni x_1$, $U_2 \ni x_2$ such that $U_1 \cap U_2$ is empty.
If $X$ is a Hausdorff space, we also say that $X$ satisfies \dfn{Axiom $T^2$}.
\end{definition}

There are other properties called Axiom $T^s$, where if $s' \leq s$ then $T^s$ implies $T^{s'}$. For example:

\begin{definition}
Let $X$ be a topological space. We say that $X$ satisfies:
\begin{enumerate}
\item \dfn{Axiom $T^0$}, if for every $x_1, x_2 \in X$, either $x_1 = x_2$ or there is an open set $U$ such that $x_1 \in U$ and $x_2 \notin U$.
\item \dfn{Axiom $T^1$}, if for every $x \in X$, $\{x\}$ is a closed set.
\item \dfn{Axiom $T^4$}, if for every two closed sets $K_1, K_2 \subseteq X$, either $K_1 = K_2$ or there are open sets $U_1, U_2$ such that $K_1 \subseteq U_1$, $K_2 \subseteq U_2$, and $U_1 \cap U_2$ is empty.
\end{enumerate}
Axioms of the form Axiom $T^s$ are called \dfn{separation axioms}.
\end{definition}

\begin{lemma}
One has a string of implications: Axiom $T^4$ implies Axiom $T^2$, which implies Axiom $T^1$, which implies Axiom $T^0$.
\end{lemma}

\begin{lemma}
Let $X$ be a semimetric space. The following are equivalent:
\begin{enumerate}
\item $X$ is a metric space.
\item $X$ satisfies Axiom $T^4$.
\item $X$ satisfies Axiom $T^0$.
\end{enumerate}
Moreover, if $R$ is the equivalence relation defined by $x_1 R x_2$ iff $d(x_1, x_2) = 0$, then $X/R$ can be given the structure of a metric space.
\end{lemma}

In particular, every metric space is Hausdorff and every semimetric space can be turned into a metric space by identifying points that are indistinguishable by its topology.

Finally let us discuss compactness.

\begin{definition}
An \dfn{open cover} $\mathcal U$ of a topological space $X$ is a set of open sets such that $\bigcup \mathcal U = X$.
A \dfn{subcover} of $\mathcal U$ is an open cover $\subseteq \mathcal U$.
\end{definition}

\begin{definition}
A \dfn{compact space} is a topological space such that every open cover has a finite subcover.
A \dfn{locally compact space} is a topological space $X$ such that for every $x \in X$ there is an open set $U \ni x$ such that the smallest closed set containing $U$ is compact.
\end{definition}

\begin{lemma}
Let $X$ be a topological space. Then $X$ is compact iff for every net $x$ in $X$, there is a subnet $x'$ such that $x'$ is convergent.
Moreover, if $X$ is first-countable, then we can replace the word ``net" by ``sequence" in the previous sentence.
\end{lemma}

\begin{lemma}
Let $X$ be a topological space. Then:
\begin{enumerate}
\item If $X$ is Hausdorff and $K \subseteq X$ is compact, then $K$ is closed.
\item If $f: X \to Y$ is continuous and $X$ is compact, then $f(X)$ is compact.
\end{enumerate}
In particular, if $X$ is compact, $Y$ is Hausdorff, and $f: X \to Y$ is continuous, then $f$ sends closed sets to closed sets.
\end{lemma}

\begin{theorem}[Heine-Borel]
$\RR^d$ is locally compact, and $K \subseteq \RR^d$ is compact iff $K$ is closed and bounded.
\end{theorem}

\begin{lemma}
If $K$ is a compact space, then $K$ is Hausdorff iff $K$ satisfies Axiom $T^4$.
\end{lemma}

We now arrive at the two nontrivial theorems of point-set topology which will be used in the main text as black boxes.
Since they are so important we sketch their proofs, but since these results rely on the above lemmata which we have not proved, these arguments are incomplete; see TODO:Cite for a complete exposition.

\begin{theorem}[Urysohn lemma and Tietze extension]
\label{Urysohn and Tietze}
Let $X$ be a topological space, $A, K \subseteq X$ are disjoint, and $A$ is closed. Suppose that one of the following is true.
\begin{enumerate}
\item $X$ satisfies Axiom $T^4$ and $K$ is closed.
\item $X$ is a locally compact Hausdorff space and $A$ is compact.
\end{enumerate}
Then:
\begin{enumerate}
\item There is a continuous function $f: X \to [0, 1]$ such that $f|A = 0$ and $f|K = 1$.
\item For every interval $L \subseteq \RR$ and every continuous function $g: K \to L$, there is a continuous function $G: X \to L$ such that $G|K = g$.
\end{enumerate}
\end{theorem}
\begin{proof}
We prove this assuming Axiom $T^4$. The proof assuming that $X$ is a locally compact Hausdorff space follows from the case of Axiom $T^4$ and the previous lemma after a little work, which we omit.

We now prove the existence of $f$.
Let $\overline U$ denote the smallest closed set containing an open set $U$.
Set $U(1/2)$ to be an open set containing $K$ which is disjoint from a neighborhood of $A$, which exists by Axiom $T^4$.
Then $U(1/2)^c$ is closed and so by Axiom $T^4$ we can find $U(1/4)$ open such that $\overline{U(1/4)} \subseteq U(1/2)$.
Similarly we can find $U(3/4)$ open such that $\overline{V(1/2)}$ is contained in the interior of $U(3/4)$.
Repeating this process we define $U(p/2^n)$ for every $0 < p < 2^n$, $n \in \NN$, such that $\overline{U(s)} \subseteq U(r)$.
Now set
$$f(x) = \inf_{x \in U(r)} r$$
if such an $r$ exists, otherwise $f(x) = 1$.
Since $\{p/2^n: 0 < p < 2^n, n \in \NN\}$ is dense in $[0, 1]$, one can check that $f$ is continuous, so $f$ is as desired.

We may assume that $L = [0, 1]$; we omit the details.
We may then set $G|A = 0$, and use the function $f$ to interpolate between $A,K$. We omit the details.
\end{proof}

Note carefully that the above result does not imply that every locally compact Hausdorff space satisfies Axiom $T^4$ -- in fact this is false.

\begin{theorem}[Tychonoff]
Let $\mathcal X$ be a nonempty set of nonempty compact spaces. Let $X = \prod \mathcal X$. Then $X$ is compact and nonempty.
\end{theorem}
\begin{proof}
By Zermelo's well-ordering theorem we may write $\mathcal X = (X_\alpha)_{\alpha < \kappa}$ where $\kappa$ is a cardinal and the $\alpha$ are ordinals.
We proceed by transfinite induction, since $\kappa$ can be viewed as an ordinal.
The base case is $\kappa = 1$, in which case $X = X_0$ is obviously compact.

The inductive case is that $\kappa = \beta + 1$, where $Y = \prod_{\alpha < \beta} X_\alpha$ is compact and $X_\beta$ is compact.
So we must show that $X = Y \times X_\beta$ is compact.
Let $x = (y, z)$ be a net in $X$, where $y$ is a net in $Y$ and $z$ is a net in $X_\beta$.
By compactness there is a subnet $x' = (y', z')$ such that $y'$ is a convergent net in $Y$.
Then by compactness again there is a subnet $x'' = (y'', z'')$ of $x'$ such that $z''$ is a convergent net in $X_\beta$.
But $y'$ is convergent in $Y$ and $y''$ is a subnet of $y'$ so $y''$ is convergent in $Y$.
Therefore $x''$ is convergent in $X$.

The interesting case is the limit case, thus if $\delta < \kappa$ then $Y_\delta = \prod_{\alpha < \delta}$ is compact, and $\kappa$ is not the successor of any ordinal.
In that case $X = \prod_{\alpha < \kappa} X_\alpha$, and if $x$ is a net in $X$ then we can let $y^\delta$ be the projection of $x$ onto $Y_\delta$.
So we can find subnets $z^\delta$ of $y^\delta$ which converge.

Let $\PP$ be the domain of $x$. We may assume that $\PP = \gamma$ for some cardinal $\gamma$, by the lemma:
\begin{lemma}
Let $Q$ be a partially ordered set. Then there is a cardinal $\gamma$ and an injection $f: Q \to \gamma$ such that if $x \leq y$ then $f(x) \leq f(y)$.
\end{lemma}
This is a straightforward consequence of Zorn's lemma, once we use Zermelo's well-ordering theorem to find $\gamma > \card Q$ large enough.
We omit the details.

By replacing $\kappa$ with a larger cardinal $\lambda$, and setting $X_\beta$ to be a point for all $\kappa \leq \beta < \lambda$, we may assume that $\gamma \leq \kappa$.
For every $\alpha$ let $x_\alpha' = z^\alpha_\alpha$.
Then $x'$ converges since $z^\alpha$ does for every $\alpha$.
So $x'$ has a limit. Therefore $X$ is compact.
\end{proof}

Note that the above use of the axiom of choice (as Zermelo's well-ordering theorem and Zorn's lemma) cannot be avoided. There is a theorem of Kelley which says that if we are in Zermelo-Fraenkel set theory possibly without the axiom of choice, but Tychonoff's theorem holds, then so does the axiom of choice.
