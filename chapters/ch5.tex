\chapter{Differentiation and regularity}


\section{Differentiation of measures}

As an application, we define the conditional expectation of a random variable.
Recall from elementary probability theory that if $A$ is an event (which is not almost surely false) and $X$ is a random variable, then the conditional expectation of $X$ given $A$ is
\begin{equation}\label{conditional expectation event}
E(X|A) = \frac{E(1_AX)}{P(A)},
\end{equation}
the expected value of $X$ if we rescale $A$ to be the entire probability space.
However, in probability theory, one often needs to discuss the conditional expectation of $X$ with respect to not just an event, but an entire $\sigma$-algebra of events.
More precisely, let $\mathcal F$ be a $\sigma$-algebra.
If $A \in \mathcal F$ and we define $E(X|A)$ by (\ref{conditional expectation event}), then we forget everything $X$ except its mean on $A$.
The conditional expectation of $X$ on $\mathcal F$, by definition, will be a random variable that only remembers the conditional expectations of $X$ with respect to every event in $\mathcal F$.

\begin{definition}
Let $(\Omega, \Sigma, P)$ be a probability space, $\mathcal F \subseteq \Sigma$ a $\sigma$-algebra, and $X \in L^1(\Omega \to \RR)$ a random variable.
The \dfn{conditional expectation} of $X$ given $\mathcal F$ is a measurable function
\[E(X|\mathcal F): (\Omega, \mathcal F) \to \RR\]
such that
\[E(1_{A} E(X|\mathcal F)) = E(1_{A} X)\]
whenever $A \in \mathcal F$.
\end{definition}

\begin{example}
A \dfn{measurable partition} of $\Omega$ is a set of mutually exclusive events $A_i$ such that $\bigcup_{i} A_{i} = \Omega$.
If $(A_i)$ is a measurable partition and $\mathcal F$ is the smallest $\sigma$-algebra containing every $A_i$, then $E(X|\mathcal F)$ is constant on each $A_i$, namely
\[E(X|\mathcal F)|A_{i} = E(1_{A_i} X)\]
is the mean of $X$ on $A_i$.
\end{example}

\begin{corollary}
Let $(\Omega, \Sigma, P)$ be a probability space, $\mathcal F \subseteq \Sigma$ a $\sigma$-algebra, and $X \in L^1(\Omega \to \RR)$ a random variable.
The conditional expectation $E(X|\mathcal F)$ is well-defined in the sense that it exists, and if $Y$ is also a conditional expectation, then $E(X|\mathcal F) = Y$ almost surely.
\end{corollary}
\begin{proof}
Let $\mu$ be the measure
\[\mu(A) = E(1_{A} X)\]
defined for $A \in \mathcal F$.
Then $\mu$ is absolutely continuous with respect to $P$, so it has a Radon-Nikodym derivative; we set
\[E(X|\mathcal F) = \frac{d\mu}{dP}.\]
Then
\[E(1_{A} E(X|\mathcal F)) = \int_{A} \frac{d\mu}{dP} ~dP = \int_{A} ~d\mu = \mu(A) = E(1_{A} X)\]
as desired.

For the uniqueness, assume $Y$ is also a conditional expectation and let $A$ be the event that $Y \neq E(X|\mathcal F)$.
Then $A \in \mathcal F$ so
\[E(1_A|Y - E(X|\mathcal F)|) = E(1_A|X - X|) = 0.\]
But $|Y - E(X|\mathcal F)| > 0$ on $A$ so this implies $E(1_A) = 0$, thus $A$ is almost surely false.
\end{proof}

\begin{exercise}
Let $X \in L^1$ be a random variable and $\mathcal F$ a $\sigma$-algebra. Show that if the pullback $\sigma$-algebra induced by $X$ is independent of $\mathcal F$ then
\[E(X|\mathcal F) = E(X).\]
Show that
\[E(E(X|\mathcal F)) = E(X).\]
Show that if $X$ is $\mathcal F$-measurable then $E(X|\mathcal F) = X$.
\end{exercise}

\begin{exercise}
Verify that the monotone and dominated convergence theorems, and Fatou lemma, are valid when $EX_{n}$ is replaced with $E(X_{n}|\mathcal F)$, $\mathcal F$ a $\sigma$-algebra.
\end{exercise}

\section{Existence of Radon measures}

\section{Differentation of vector-valued functions}

\section{Vitali's covering lemma}

\begin{definition}
If $B$ is a ball, say $B = B(x, r)$, we let $kB$ denote the \dfn{dilated ball} $kB = B(x, kr)$.
\end{definition}
\begin{lemma}
Let $X$ be a metric space and let $\mathcal B$ be a finite set of balls in $X$. Then there is a subset $\mathcal B_0$ of $\mathcal B$ consisting of disjoint balls such that
\[\bigcup_{B \in \mathcal B} B \subseteq \bigcup_{B_0 \in \mathcal B_0} 3B_0.\]
\end{lemma}
TODO prove me
\begin{theorem}[Vitali's covering lemma]
Let $X$ be a separable metric space and let $\mathcal B$ be a set of balls in $X$. Let
\[R(\mathcal B) = \sup_{B(x, r) \in \mathcal B} r\]
be the supremum of radii of the balls in $\mathcal B$. If $R(\mathcal B) < \infty$ then for every $\varepsilon > 0$ there is a countable subset $\mathcal B_0(\varepsilon)$ of $\mathcal B$ consisting of disjoint balls such that
\begin{equation}\label{Vitali formula}
\bigcup_{B \in \mathcal B} B \subseteq \bigcup_{B_0 \in \mathcal B_0(\varepsilon)} (3 + \varepsilon)B_0.
\end{equation}
\end{theorem}
TODO Prove me

\begin{exercise}
Show that there is a counterexample to Vitali's covering lemma when $R(\mathcal B)$ is infinite.
\end{exercise}

\begin{exercise}
Show that a version of Vitali's covering lemma holds for any metric space.
That is, show that if $X$ is a metric space and $\mathcal B$ is a set of balls in $X$ such that $R(\mathcal B) < \infty$, then for every $\varepsilon > 0$ there is a subset (not necessarily countable) $\mathcal B_0(\varepsilon)$ of $\mathcal B$ consisting of disjoint balls such that (\ref{Vitali formula}) holds.
You will need to use some kind of set-theoretic machinery to pull this off.
\end{exercise}

\section{The maximal inequality}

\begin{theorem}[weaktype Hardy-Littlewood maximal inequality]
Let $d \geq 1$ and $f \in L^1(\RR^d)$. Let $\mu$ denote Lebesgue measure. Then for every $\lambda > 0$,
\[\mu(\{Mf > \lambda\}) \leq 3^{d} \frac{||f||_{L^1(\RR^d)}}{\lambda}.\]
\end{theorem}
TODO prove me

\begin{exercise}
Let $f \in L^1(\RR^d)$.
Show that if $Mf \in L^1(\RR^d)$ then $f = 0$.
It may help to first try this when $f$ has compact support.
\end{exercise}

\section{The Lebesgue differentation theorem}

\section{The fundamental theorem of calculus}
